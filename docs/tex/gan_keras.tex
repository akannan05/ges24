\documentclass[a4paper]{article}

%import package for math and image
\usepackage{graphicx}
\usepackage{tcolorbox}
\usepackage{amsmath}
\usepackage{amssymb}

%english
\usepackage[english]{babel}
\usepackage[utf8]{inputenc}

%for code
\usepackage{listings}

\title{GANs}

\author{Ani}

\begin{document}
	
	\maketitle
	
	\section{Intro to GANs}
	
	\textbf{Remark:} Note that GANs stand for Generative Adversarial Networks.
	
	\subsection{Generative Models:}
	
	\begin{tcolorbox}
		\textbf{Definition 1.1:} \\
		
		A generative model \emph{generates} new data samples that looks like the training data.
	\end{tcolorbox}
	
	\begin{tcolorbox}
		\textbf{Example 1.2:} \\
		
		Consider a coin flip generative model. \\
		
	\end{tcolorbox}
	
	\subsection{Adversarial Networks:}
	
	\subsection{Neural Networks:}
	
	\section{Math and Theory of GANs}
	
	\section{Implementation of GANs in Keras}
	
\end{document}
